\chapter{Concetti di base}

\section*{Introduzione}
Qui illustriamo alcuni dei concetti di base che serviranno per comprendere il resto della discussione, daremo una definizione di ontologia e tesauro, descriveremo brevemente gli strumenti utilizzati e i linguaggi con cui si descrivono le basi di conoscenza che tratteremo.


\section{Ontologie e tesauri}

\subsection{ontologie}
\subsubsection{Esempio}
Consideriamo una semplice ontologia che rappresenta persone con legami di parentela genitore-figlio; le persone hanno uno o più nomi salvate nel tag \verb|comment|. Modelliamo questa ontologia con una classe \verb|Persone| e una sottoclasse \verb|Genitori| (i cui individui sono \verb|Persone| che realizzano la relazione \verb|genitoreDi|). Creiamo la relazione \verb|genitoreDi|. Infine popoliamo l'ontologia con alcuni individui. Il risultato ottenuto con Protégé è un documento XML di questo tipo:
\addxml{persone.rdf}{lst:persone.rdf}{Code/persone.rdf}

\subsection{Tesauri}

\section{Strumenti di editing}

\subsection{Protégé}

\subsection{\cduce}
\cduce è un linguaggio di programmazione funzionale, staticamente tipato e orientato allo sviluppo di applicazioni che lavorano su documenti XML\cite{cduceLanguage}
\subsection{Feature}\label{fature_cduce}
\subsubsection{Pattern matching}
\label{CDucePattern}
È un'operazione fondamentale in \cduce ed ha la forma:
\begin{minted}[tabsize=2, breaklines, bgcolor=bg]{OCaml}
match e with
	| p1 -> e1
	...
	| pn -> en
\end{minted}
Si cerca di fare il match tra la valutazione di un'espressione \verb|e| e vari pattern \verb|pi|. Il primo pattern che fa il match con \verb|e| attiva la corrispondente espressione sulla destra che può usare le variabili legate dal pattern.


\section{Metalinguaggi}

\subsection{OWL}

\subsection{SKOS}
