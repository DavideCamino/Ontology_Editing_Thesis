\chapter{Conclusioni}
\section{Maturità di \cduce}
Prima di utilizzare un software in applicazioni importanti è bene domandarsi se lo stesso sia affidabile e vanga manutenuto; questo al fine di evitare situazioni in cui tutto l'applicativo costruito al di sopra di uno strumento crolli al crollare delle fondamenta. Vediamo quindi qual è lo stato di sviluppo di \cduce e quali sono stati le difficoltà tecniche dell'utilizzarlo.
\subsection{Il progetto \cduce}
Come si può leggere nella loro pagina GitLab\footnote{https://gitlab.math.univ-paris-diderot.fr/cduce/cduce} il progetto nasce e si sviluppa come una ricerca condivisa tra il gruppo di ricerca sui linguaggi all'ENS\footnote{https://www.ens.psl.eu/} di parigi e il gruppo di ricerca sui Database all'LRI\footnote{https://www.lri.fr/} ad Orsay. Attualmente il progetto è mantenuto dai PPS laboratory\footnote{https://www.irif.fr/} e dal  Toccata Group\footnote{https://toccata.gitlabpages.inria.fr/toccata/}.

Lo scopo del progetto è quello di costruire un linguaggio per manipolare documenti XML che usi seriamente i tipi XML; questo porta i vantaggi descritti in \ref{fature_cduce}. L'implementazione attuale ha lo scopo di mostrare proprio queste feature innovative e di validare le scelte di progettazione effettuate.
\subsection{Stato di sviluppo attuale}
Sulla pagina del progetto\footnote{https://www.cduce.org/} le ultime informazioni risalgono al 2021 e spiegano che il gli sviluppatori si stanno impegnando in una riscrittura completa del compilatore per separare la libreria che gestisce i tipi dal resto del compilatore. La riscrittura del compilatore potrebbe provocare inconsistenza con le ultime versioni di OCaml pertanto si descrive un modo per aggirare il problema oppure si consiglia di utilizzare la vecchia versione. Nonostante le ulime informazioni sul sito siano del 2021 controllando la pagina GitLab si vede che, anche se il branch \say{stable} non riceve aggiornamenti da un anno, altri branch sono tutt'ora attivi\footnote{alla data in cui si scrive (10/10/2023) l'ultima attività risulta del 09/10/203}.
\subsection{Difficoltà incontrate}
Ho testato \cduce su una distribuzione Ubuntu\footnote{https://ubuntu.com/desktop} e su Arch\footnote{i use Arch btw: https://archlinux.org/}; su Ubuntu non ho avuto successo (seguono dettagli), quindi tutte le prove descritte sono state fatte su Arch.
\subsubsection{Ottenere \cduce}
Sulla pagina del progetto si legge che \cduce è pacchettizzato per le più importanti distribuzioni linux, ma questi pacchetti, almeno per Ubuntu risalgono a versioni molto vecchie (Ubuntu 12.10 e 13.04) e non funzionano nelle distribuzioni attuali.

Seguendo le istruzioni per compilare la nuova versione (sia polimorfa che monomorfa) di \cduce il processo fallisce, probabilmente questo è dovuto al fatto che da quando è stata scritta la guida ad oggi i pacchetti di OCaml sono stati ulteriormente aggiornati, e il workaround descritto non è più sufficiente. Ho provato, questa volta con successo, ad installare la vecchia versione di \cduce (version 0.6.1-rc1), ho scelto la versione monomorfa perché la versione polimorfa veniva descritta come \say{buggy and experimental}

Il processo di installazione descritto sul sito consiste a grandi linee nel:
\begin{itemize}
	\item creare un ambiente virtuale per OCaml;
	\item installare in questo ambiente le vecchie versioni dei pacchetti necessari a compilare \cduce;
	\item compilare i sorgenti;
	\item ottenere l'eseguibile di \cduce.
\end{itemize}
\subsubsection{Reperire informazioni}
Sul sito si possono trovare molte risorse utilissime per imparare ad usare \cduce, sono presenti in particolare 2 documenti:
\begin{itemize}
	\item una guida utente\footnote{https://www.cduce.org/manual.html} che spiega in modo molto rigoroso l'approccio del linguaggio ai vari tipi, operazioni e funzioni. La spiegazione è molto tecnica, presenta ogni argomento nel modo più generale possibile facendo estensivo uso dei pattern, per leggerla è necessaria qualche base sui linguaggi funzionali e su come il pattern matching operi (soprattutto se si intendono consultare solo alcune sezioni);
	\item un tutorial\footnote{https://www.cduce.org/tutorial.html} che parte direttamente da degli esempi per spiegare il modo in cui \cduce opera. Gli esempi presentati sono molto ben curati e commentati, e permettono di cominciare subito a sperimentare con lo strumento.
\end{itemize}
Entrambe le guide hanno una controparte in PDF molto comoda per poter cercare all'interno dell'intero manuale un certo termine.

Le difficoltà sorgono nel momento in cui siamo interessati a certe sezioni del tutorial, alcune parti mancano per intero, ed essendo uno strumento relativamente di nicchia non c'è un forum o una community estesa alla quale si possa ricorrere per domande o dubbi. Questo rende particolarmente difficile capire certe parti della guida utente che senza esempi risultano particolarmente oscure. In particolare durante gli esperimenti eseguiti è stato importante fare pattern sulle sequenze, questa sezione (come le precedenti di quel capitolo) è assente nel tutorial. Avendo delle basi di Haskell\footnote{https://www.haskell.org/} e cercando di interpretare la guida che è piuttosto criptica in quel capitolo si può provare a immaginare quale sia la struttura corretta per fare il match (è servito nelle funzioni \verb|andList| e \verb|orList| in \ref{lst:utility.cd}).
\subsubsection{Output}
Nei linguaggi funzionali le operazioni di input e output sono sempre delicate, in \cduce per poter esportare un documento XML dobbiamo fare 2 passi: prima trasformiamo l'elemento di tipo \verb|AnyXml| (che contiene il file XML che vogliamo salvate) in una stringa poi facciamo il dump della stringa su file. Per la conversione da XML a stringa abbiamo 2 possibilità, se il documento contiene solo caratteri previsti dalla norma  ISO-8859-1\footnote{https://www.iso.org/standard/28245.html} si può usare \verb|print_xml| altrimenti per preservare tutti i caratteri si usa \verb|print_xml_utf8|. Ottenuta la stringa si può fare il dump su file con \verb|dump_to_file| oppure \verb|dump_to_file_utf8|, questo sarà un file XML che nel nostro caso rappresenta un ontologia.

Provando ad aprire il risultato di un dump con un editor di testo ci si tende conto che tutto il codice XML si trova su una sola riga, manca completamente qualsiasi formattazione e il file risulta dunque illeggibile. Aprendo il file con uno strumento come Protégé non ci sono problemi, il file è ben formato e viene interpetato correttamente, ma se vogliamo vedere il risultato della manipolazione con \cduce senza ricorrere ad altri software appositi ci troviamo in difficoltà\footnote{senza contare che Protégé quando si salva il file sul quale si ha lavorato sì lo riformatta correttamente, ma riposiziona gli elementi e aggiunge parti non presenti in origine (anche semplicemente i commenti) rendendo imossibile conoscere il documento originale}. Se siamo intenzionati a vedere il risultato del dump di \cduce dobbiamo riformattare il documento per renderlo fruibile. Per fare questo ci sono molte opzioni, in questo caso è stato sviluppato un piccolo programma in java che riformatta il file XML rendendo possibile un'ispezione dello stesso mediante un editor di testo. 

Il problema è probabilmente dovuto al fatto che le funzioni di \verb|pretty-printing| di \cduce usano per le operazioni di output funzioni di OCaml, avendo forzato l'allineamento tra i pacchetti di OCaml e \cduce è possibile che non sia tutto esattamente compatibile, questo fa sì che le componenti non funzionali di \cduce e in generale l'interazione tra \cduce e OCaml non funzioni correttamente.

È in ogni caso un peccato che attualmente l'unico modo per poter leggere il risultato dell'esecuzione di un programma scritto con un linguaggio funzionale, usato senza ricorrere a nessuno stratagemma imperativo richieda però un programma in java per essere fruito.
\subsection{Sviluppi futuri}
Considerando che il progetto sembra essere ancora di interesse per i gruppi di ricerca che vi hanno lavorato e che il repository risulta attivo ed utilizzato è auspicabile che la riscrittura del compilatore iniziata nel 2021 termini e la nuova versione di \cduce sia polimorfa e sfrutti l'ultima versione di OCaml, questo ridurrebbe drasticamente i problemi descritti sopra e potrebbe spingere più persone ad interessarsi al progetto e a collaborare per rendere \cduce uno strumento professionale a tutti gli effetti; attualmente \cduce viene considerato dagli sviluppatori stessi come un prototipo di ricerca e non adatto ad applicazioni stabili\footnote{https://gitlab.math.univ-paris-diderot.fr/cduce/cduce/-/wikis/home}.
\section{Punti di forza di \cduce}
Nonostante le difficoltà incontrate \cduce si è rivelato uno strumento molto potente e versatile per la manipolazione di ontologie, in particolare il fatto di essere un linguaggio funzionale lo rende particolarmente sintetico e leggibile una volta presa dimestichezza coi concetti base.
\subsection{Sistema di tipi}
I tipi descrivono un insieme di valori costruiti in un certo modo, è estremamente facile definire un tipo e la sua struttura, d'altra parte avere dei tipi definiti correttamente ci permette di scrivere funzioni che li manipolano come ci aspettiamo e che restituiscono un elemento esattamente del tipo che vogliamo.

Tutta la verifica dei tipi in \cduce avviene staticamente, saremo quindi sicuri che una funzione trasformi un concetto di un tesauro in una classe di un'ontologia ancora prima di eseguire questa funzione perché \cduce verifica che tutte le trasformazioni che applichiamo permettano di passare da un elemento del primo tipo ad un elemento del secondo.

Il controllo di tipo avviene per tutte le funzioni che definiamo (a patto di specificarne correttamente l'interfaccia) e permette di scrivere del codice in cui la maggior parte del debug può essere fatta in fase compilazione e non a run-time. 

\subsection{Funzioni di ordine superiore}
Come tutti i linguaggi funzionali anche \cduce permette di definire funzioni di ordine superiore, queste hanno numerosi vantaggi: 
\begin{itemize}
	\item possiamo scrivere funzioni semplici di cui è facile verificare la correttezza e assemblarle in funzioni più articolate che diventano più trattabili, gestendo in questo modo la complessità;
	\item le funzioni di ordine superiore permettono di generalizzare il codice in modo estremamente efficace, consideriamo il caso di una funzione che rimuove alcuni elementi di una lista secondo un certo criterio, nel momento in cui cambia il criterio dovremmo riscrivere almeno parte della funzione, se invece implementiamo un una funzione per la rimozione selettiva che fra i vari parametri accetta una funzione che specifica se un elemento della lista va eliminato possiamo scrivere una funzione assolutamente generale che possa lavorare con un qualsiasi criterio che andremo volta per volta a specificare. Su questa possibilità torneremo dopo in \ref{ch5.4_func}
\end{itemize}
\subsection{Pattern matching}
Grazie al pattern matching è possibile effettuare complesse operazioni di estrazione dei dati e manipolazione degli stessi, è un aspetto centrale di \cduce e ne incrementa moltissimo la potenza espressiva. Purtroppo non è lo strumento più semplice da padroneggiare e ad una prima vista può sembrare ambiguo il modo in cui descrivere un certo pattern. Lievi differenze nella definizione del pattern producono effetti differenti, e bisogna prestare particolare attenzione a descrizioni che apparentemente possono sembrare equivalenti.
\section{Spunti per paragoni futuri}
In questo lavoro abbiamo provato a fare dei paragoni tra \cduce e Protégé, sono due strumenti molto differenti e le scale di paragone per forza, a seconda dei parametri considerati, fanno risaltare uno strumento come molto efficace e l'altro come inappropriato. Potrebbe essere quindi interessante confrontare \cduce con altri strumenti più simili.
\subsection{Confronto con un linguaggio imperativo}
Esistono vari linguaggi che incorporano librerie per la modellazione di documenti XML, uno fra tutti è sicuramente java.
\subsubsection{JAXP}
JAXP\cite{JAXP} è l'API che java mette a disposizione per processare dati in XML. I packages contenuti in questa API consentodo alle applicazioni di fare il parsing, trasformare, validare e interrogare con delle query documenti XML. JASP è dotata di un layer che permetta l'indipendenza tra il codice dell'applicazione e il particolare processor XML implementato. JAXP, rispetto a \cduce che è attualmente in stato di sviluppo ed è considerato un prototipo, garantisce la possibilità di sviluppare applicazioni per il processing di file XML che siano perfettamente funzionanti, integrabili e stabili.

Alla luce dei vantaggi di affidabilità offerti da progetti molto più grandi di \cduce potrebbe essere interessante valutare come l'uso di un linguaggio funzionale possa apportare benefici a livello di programmazione, correttezza del codice e leggibilità rispetto ad un linguaggio essenzialmente imperativo come java.
\subsection{Confronto con un linguaggio funzionale}
\cduce nasce con lo scopo di processare documenti XML, esistono altri linguaggi più general porpouse che implementano la capacità di elaborare XML tramite librerie. Uno dei linguaggi che presenta ampia scelta di librerie è Haskell. Questo linguaggio è sicuramente più noto di \cduce e sono disponibili, pertanto, numerose risorse online per chiarire domande e dubbi che possono sorgere. Fra le librerie per l'elaborazione di documenti XML citiamo HaXml\footnote{https://archives.haskell.org/projects.haskell.org/HaXml/} e Haskell XML Toolbox (HXT)\footnote{https://wiki.haskell.org/HXT}, che si basa sulle idee di HaXml, entrambi i progetti risultano attivi e documentati.

Il vantaggio di usare una di queste librerie è la possibilità di interfacciarsi direttamente con Haskell potendo integrare le funzionalità della libreria scelta con le numerosissime funzioni offerte da Haskell (sia prese dai pacchetti base sia implementate da librerie esterne). Anche \cduce dovrebbe fornire integrazione con OCaml, sia permettendo di richiamare OCaml in \cduce sia viceversa; queste funzionalità attualmente, almeno sulla base delle prove effettuate, non funzionano.
\subsection{Confronti possibili}
Abbiamo presentato altri strumenti alternativi a \cduce, i confronti possibili potrebbero basarsi sulle seguenti caratteristiche
\begin{itemize}
	\item velocità di sviluppo del codice: ci aspettiamo che un linguaggio funzionale possa presentare dei vantaggi soprattutto nell'esprimere delle trasformazioni degli elementi rispetto ad un linguaggio imperativo; sfruttando poi la funzione \verb|map| (sia in \cduce che Haskell) risulta anche immediato appplicare la trasformazione ad intere liste di elementi;
	\item correttezza del software: anche in questo caso un linguaggio funzionale che esegua un controllo statico dei tipi in fase di compilazione può aiutarci ad evidenziare errori prima che si presentino a run-time;
	\item scalabilità e integrazione: probabilmente da questo punto di vista un linguaggio come java offre dei vantaggi essendo pensato per anche per lo sviluppo di applicativi di grandi dimensioni più che Haskell o \cduce (in ogni caso nulla vieta di integrare \cduce in applicazioni scritte in altri linguaggi, questa possibilità è approfondita in \ref{ch5.4});
	\item  velocità di esecuzione: durante gli esperimenti si è notato un drastico rallentamento dell'esecuzione quando si usavano funzioni ricorsive per creare gli insiemi di sottoclassi, può essere utile provare almeno in java ad implementare una versione iterativa dello stesso algoritmo per verificare se le prestazioni cambiano significativamente. Per quanto riguarda le query può essere interessante un confronto dato che queste ultime in \cduce sono ottimizzate con le stesse tecniche di ottimizzazione della logica classica SQL.
\end{itemize}
\section{Sviluppo di nuovi strumenti}
\subsection{Criticità di \cduce}
\cduce si è rivelato essere un valido e potente strumento per la manipolazione di file XML; a prescindere dai problemi riscontrati durante gli esperimenti, che auspicabilmente verranno tutti risolti con il rilascio della nova versione stabile, la più grande difficoltà che un utente incontra quando si approccia all'uso di questo strumento è proprio la difficoltà di esprimere degli algoritmi con un linguaggio funzionale.

Se chi vuole usare \cduce per manipolare delle ontologie non ha grandi competenze in campo di programmazione ed ha usato fino a quel momento solo strumenti grafici per la creazione o manipolazione di ontologie si scontra con una notevole difficoltà di sviluppo degli strumenti, che solo in parte sono guidate dal tutorial presente sul sito, in numerosi casi bisogna essere in grado di interpretare la guida che ad un utente non esperto può sembrare particolarmente oscura.

Discorso analogo vale per un utente già competente nell'uso di linguaggi imperativi che si trova a dover fare i conti con un nuovo paradigma di programmazione senza che la guida o il tutorial aiutino particolarmente. Probabilmente un utente di questo tipo si troverà ancora più confuso di un utente novizio alla programmazione.

\cduce presuppone che i suoi utenti posseggano delle competenze già abbastanza avanzate nell'uso di linguaggi funzionali, che si trovino a proprio agio nello sviluppare algoritmi ricorsivi e ad usare intensivamente il pattern matching. Date le competenze richieste e la curva d'apprendimento particolarmente ripida \cduce potrebbe rivelarsi uno strumento ostico per la maggior parte degli utenti che vogliono concentrarsi sulla creazione di un'ontologia che sia corretta e ben formata e non sullo sviluppo di codice per la creazione stessa.

\subsection{Interfaccia grafica}\label{ch5.4}
Fatte le precedneti considerazioni ne risulta che nonostante \cduce sia uno strumento potente ed espressivo sia anche particolarmente poco fruibile. Per arginare la difficoltà d'uso si potrebbe pensare ad un interfaccia che permetta di rendere grafiche la maggior parte delle operazioni, relegando la parte di programmazione a task specifici o molto puntuali sulla particolare ontologia che si sta manipolando.

Esistono numerose interfacce da cui trarre ispirazione (l'interfaccia di Protégé per citarne una) e \cduce si presta particolarmente bene a creare codice generale:
\begin{itemize}
	\item parsing di una generica ontologia: come si può notare in \ref{lst:general_structure} la struttura definita permette di fare correttamente il parsing di tre ontologie completamente differenti, la struttura non è esaustiva e si potrebbero dover aggiungere altre definizioni di tipi per renderla completamente generale senza rinunciare ai controlli di correttezza\footnote{se si usasse il tipo \texttt{AnyXml} la struttura sarebbe generalissima ma non offrirebbe più controlli puntuali sulla correttezza dei tipi ritornati dalle funzioni};
	\item funzioni generali per trattare con ontologie: guardando il codice in \ref{lst:utility.cd} ci si rende conto che queste funzioni non hanno alcuna attinenza con le particolari ontologie che stiamo trattando, un parco sufficientemente ampio di queste funzioni sarebbe in grado di estrarre la maggior parte delle informazioni utili da un ontologia, di selezionare e di assemblare le parti che l'utente vorrebbe poter mantenere;
	\phantomsection
	\item\label{ch5.4_func}funzioni di ordine superiore: laddove le funzioni generali non bastano si potrebbero pensare funzioni di ordine superiore complesse a piacere che servano per trattare in modo molto specifico le ontologie che si stanno trattando, queste funzioni vengono messe a disposizione dell'utente che deve solo integrare piccole e semplici funzioni che prese come argomento dalla funzione di ordine superiore la rendano specifica per la particolare ontologia d'interesse. Questo limita notevolmente la necessità di programmare dell'utente anche quando si voglia costruire degli strumenti molto specifici per la propria ontologia;
	\item creazione guidata di query: esistono già numerosi strumenti che permettono la creazione di query, anche particolarmente complesse, in modo grafico e guidato; data la profonda somiglianza tra le query espresse in linguaggio SQL e la sintassi in \cduce sarebbe possibile adattare uno degli strumenti già esistenti allo scopo. In questo modo anche le query che sono uno strumento potente ed efficiente messo a disposizione da \cduce diventerebbero facilmente accessibili.	
\end{itemize}
\section{Conclusioni}
\cduce nonostante sia classificato come prototipo è uno strumento per la manipolazione di documenti XML ed in particolare di ontologie completo ed espressivo\footnote{si fa riferimeno alla versione del software 0.6.1-rc1}. Offre una documentazione online ben curate e accessibile ad un utente con solide basi di programmazione funzionale, la documentazione è inoltre accompagnata da una serie di esempi che guidano, almeno fino ad un certo punto, un utente più novizio a muovere i primi passi con lo strumento potendo già realizzare programmi utili.

Paragonandolo ad altri strumenti si rivela acerbo\footnote{gli sviluppi sono sicuramente rallentati dal fatto di essere un progetto portato avanti da un piccolo team} e il sito riflette questa condizione essendo finito solo per metà (numerosi link e componenti non sono attualmente funzionanti). Nonostante questo, rispetto a software molto più grandi e più specificatamente orientati alla manipolazione di ontologie \cduce permette di esprimere una grande quantità di lavoro in poche righe di codice, in modo elegante e leggibile. 

Su task che richiedono l'eleborazione di molti dati, la selezione secondo certi criteri e la trasformazione seguendo certi pattern risulta molto naturale in questo linguaggio; unendo a questo il controllo statico sui tipi che garantisce la costruzione di ontologie ben formate \cduce, nonostante non sia stato pensato specificatamente per questo scopo, si dimostra uno strumento molto buono, migliore, in questi casi d'uso, di Protégé.